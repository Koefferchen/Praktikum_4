\section{Fazit} 
        
    ... Zeeman-Effekt... \\
    
    Im Versuchsteil \uproman{2}, dem Franck-Hertz-Versuch wurde die diskrete Natur quantenmechanischer Übergänge nachgewiesen. Dabei konnte gezeigt werden, dass eine Erhöhung von Ofentemperatur $T$ und Bremsspannung $U_2$ zu einer Verringerung der Messspannung $U_A$ führte, jedoch die Positionen und Abstände der Maxima unbeeinflusst blieb. Aus den Abständen der Maxima wurde schließlich die mittlere Anregungsenergie $\Delta E = \SI{4.86 +- 0.13}{eV}$ im Quecksilber-Gas bestimmt und mit den Übergängen $6^1 S_0 \rightarrow 6^3 P_0$ ($2\sigma$ Übereinstimmung) und $6^1 S_0 \rightarrow 6^3 P_1$ ($1\sigma$ Übereinstimmung) assoziiert. Die kontinuierliche Natur der Messkurve wurde auf die thermische Doppler-Verbreiterung zurückgeführ. Damit wurde der Franck-Hertz-Versuch mit \texttt{Hg} erfolgreich durchgeführt. 
