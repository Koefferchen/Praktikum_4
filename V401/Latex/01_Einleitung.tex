\section{Einleitung}
    Der nachfolgende Versuch beschäftigt sich mit Anregungszuständen in Atomen und ist dabei in zwei Teile unterteilt: \\
    
    Im Teil \uproman{1} wird der normale Zeeman-Effekt nachgewiesen, welcher zum Verständnis der magnetischen Moments eines Atoms historisch von zentraler Bedeutung war. Der Versuch wird an einer Cadmiumlampe durchgeführt, welche sich jeweils transversal oder longitudinal im Magnetfeld befindet und dessen Übergänge mithilfe eines Fabry-Perot-Etalons untersucht werden. Ziel dieses Versuchs ist der Nachweis der Übergänge $^1 D_2 \rightarrow ^1 P_1$  in Form von $\pi$, $\sigma^+$ und $\sigma^-$ Strahlung sowie deren Polarisationen.  \\
    
    Im Teil \uproman{2} wird der Frank-Hertz-Versuch mit Quecksilber (\texttt{Hg}) durchgeführt. Mit diesem belegten James Franck und Gustav Hertz die Existenz von diskreten Energieniveaus in Atomen und gewannen dafür 1925 einen Nobelpreis. In diesem Versuchs werden Elektronen durch Quecksilber-Gas beschleunigt, an welchem jene elastisch streuen. Erreichte die kinetische Energie eines Elektron jedoch die Anregungsenergie von Quecksilber, so streut jenes inelastisch und ein Einbruch im Elektronenstrom wird beobachtbar. \cite{P401_Praktikumsanleitung}
    